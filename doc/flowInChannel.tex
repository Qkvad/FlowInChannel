\documentclass[a4paper,12pt]{article}

\usepackage[utf8]{inputenc}
\usepackage[T1]{fontenc}
\usepackage{amsmath}
\usepackage{verbatim}
\usepackage{gensymb}
\usepackage{graphicx}


\newtheorem{thm}{Teorem}[section]
\newtheorem{lem}[thm]{Lema}
\newtheorem{cor}[thm]{Korolar}
\newtheorem{defn}[thm]{Definicija}
\newtheorem{rem}[thm]{Napomena}
\newtheorem{prop}[thm]{Propozicija}
\newtheorem{exa}[thm]{Primjer}
\newtheorem{conj}[thm]{Slutnja}
\newenvironment{proof}{\quad {Dokaz:}}{\hfill}
\renewcommand{\contentsname}{Sadržaj}

\begin{document}

\begin{titlepage}
    \centering
    \vspace*{\fill}
    \vspace*{0.5cm}
    \huge\bfseries
    Ubrizgavanje tekućine u vertikalni kanal
    \vspace*{0.5cm}

    \large Petra Brčić
    \vspace*{\fill}
\end{titlepage}

%\tableofcontents % show correctly after 2nd %build
%\newpage

\section{Uvod}

Ovaj rad je završni projekt za kolegij Znanstveno računanje 2. Cilj je implementirati neku od naučenih metoda i primijeniti ju za računanje rubnog problema ubrizgavanja tekućine u vertikalni kanal. 
%$3^4$ hzj
%\[ e^x  = 3 \]

%Novi redak se mora eksplicitno napisati \\ ovako

%\begin{thm}
%Neka je ovo prvi teorem. Tada uistinu i jest.
%\end{thm}
%\begin{proof}
%Trivijalno.
%\end{proof}

%\subsection{Potpoglavlje}

%\begin{defn}
%Definicijom definiramo.
%\end{defn}

\section{Problem}

Promatramo problem ubrizgavanja tekućine u jednu stranu dugog vertikalnog kanala. Jednadžbe koje opisuju ovaj problem su Navier-Stokes i jednadžbe provođenja, no one se mogu reducirati i dobivamo sljedeći sustav
\[ f''' - R [(f')^2 - ff''] + RA = 0 \]
\[ h'' + Rfh' + 1 = 0 \]
\[ \theta'' + Pf\theta' = 0 \]
\[ f(0)=f'(0)=0, f(1)=1, f'(1)=0 \]
\[ h(0)=h(1)=0 \]
\[ \theta(0)=0, \theta(1)=1 \]

Ovdje $f$ i $h$ su dvije potencijalne funkcije, $\theta$ je funkcija distribucije temperature i $A$ je nedefinirana konstanta. Dva su parametra poznatih vrijednosti, $R$ je Reynoldsov broj i $P$ je Pecletov broj (npr. $P=0.7R$).\\
  



%\begin{rem}
%Svako poglavlje ima numeraciju koja počinje s njegovim rednim brojem.
%\end{rem}

\section{Rješenje problema}

%\begin{exa}
%Primjer prethodne napomene.
%\end{exa}

Primijetimo najprije da je potproblem
\[ f''' - R [(f')^2 - ff''] + RA = 0 \]
\[ f(0)=f'(0)=0, f(1)=1, f'(1)=0 \]
odvojen od ostalih, pa ga možemo riješiti odvojeno. Tada su 
\[ h'' + Rfh' + 1 = 0 \]
\[ h(0)=h(1)=0 \]
i
\[ \theta'' + Pf\theta' = 0 \]
\[ \theta(0)=0, \theta(1)=1 \]
dva zasebna, linearna standardna problema drugog reda. Sada vidimo da smo početni problem efektivno podijelili na tri potproblema. \\

Pogledajmo sada  
\[ f''' - R [(f')^2 - ff''] + RA = 0 \]
\[ f(0)=f'(0)=0, f(1)=1, f'(1)=0  .\]
Imamo nelinearnu običnu diferencijalnu jednadžbu trećeg reda za $f$ s kontstantom $A$ koja je  određena s četiri dana početna uvjeta. Jedan način da jednadžbu dovedemo u standardnu formu je da ju deriviramo i dobijemo
\[ f''''=R[f'f'' - ff''']  .\]
Sada imamo problem zapisan u standardnom obliku 
\[ f''''=R[f'f'' - ff'''] \]
\[ f(0)=f'(0)=0, f(1)=1, f'(1)=0 \]
koji više ne uključuje $A$ eksplicitno. Jedan, još generalniji, trik je da konstantu $A$ tretiramo kao još jednu zavisnu varijablu dodavanjem obične diferencijalne jednadžbe 
\[ A'=0. \]
Problem 
\[ f''' - R [(f')^2 - ff''] + RA = 0 \]
\[ f(0)=f'(0)=0, f(1)=1, f'(1)=0 \]
\[ A'=0. \]
je sada ponovno dan u standarnoj formi. \\

Težina rješavanja nelinearnog problema je numerički ovisna, na tipičan način, o Reynoldsovom broju $R$. Za umjerenu vrijednost $R$, recimo $R=10$, prolem je jednostavan, ali postane teži povećanjem vrijednosti. Za $R=10000$ dolazi do brze promjene u nekim vrijednostima rješenja oko $x=0$. To se zove granični sloj. 

\end{document}